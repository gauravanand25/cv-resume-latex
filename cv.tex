%----------------------------------------------------------------------------------------
%	PACKAGES AND OTHER DOCUMENT CONFIGURATIONS
%----------------------------------------------------------------------------------------

\documentclass[a4paper,10pt]{article} % Default font size and paper size

\usepackage{fontspec} % For loading fonts
\defaultfontfeatures{Mapping=tex-text}
\setmainfont[SmallCapsFont = Fontin SmallCaps]{Fontin} % Main document font

\usepackage{xunicode,xltxtra,url,parskip} % Formatting packages

\usepackage{marvosym} % Allows the use of symbols

\usepackage{scrpage2} % Provides headers and footers configuration

\usepackage[usenames,dvipsnames]{xcolor} % Required for specifying custom colors

% \usepackage[big]{layaureo} % Margin formatting of the A4 page, an alternative to layaureo can be \usepackage{fullpage}
% To reduce the height of the top margin uncomment: \addtolength{\voffset}{-1.3cm}

\usepackage{hyperref} % Required for adding links	and customizing them
\definecolor{linkcolour}{rgb}{0,0.2,0.6} % Link color
\hypersetup{colorlinks,breaklinks,urlcolor=linkcolour,linkcolor=linkcolour} % Set link colors throughout the document

\usepackage{tabularx,colortbl} % Advanced table configurations
\usepackage{array} % Custom arrangement for the columns
\usepackage{multirow} %rows spanning multiple rows

\usepackage{titlesec} % Used to customize the \section command
\titleformat{\section}{\Large\scshape\raggedright}{}{0em}{}[\titlerule] % Text formatting of sections

\titlespacing{\section}{0pt}{3pt}{3pt} % Spacing around sections

\usepackage{enumitem} % For left margin removal in itemize

%----------------------------------------------------------------------------------------
% PAGE GEOMETRY
%----------------------------------------------------------------------------------------

\usepackage[top=1cm, bottom=0.5cm, lmargin=0.5cm, rmargin=1.3cm]{geometry} % The page Geometry

%----------------------------------------------------------------------------------------
% CUSTOM ENVIRONMENTS
%----------------------------------------------------------------------------------------
\newcommand{\work}[3]{
    \begin{tabular}{>{\raggedleft}p{2.2cm}|p{0.9\linewidth}}
        \textsc{#1} & \textcolor{NavyBlue}{#2}
                    \footnotesize{#3}
    \end{tabular}
}

\newcommand{\project}[4]{
    \begin{tabular}{p{0.85\linewidth}r}
        \textcolor{NavyBlue}{#2} & \multicolumn{1}{m{3cm}}{\raggedleft \textsc{#1}}\\
        #3
    \end{tabular}
    \begin{tabular}{p{\linewidth}}
        \footnotesize{#4}
    \end{tabular}
    \vspace{-0.1cm}
}

\newcommand{\lproject}[4]{
    \begin{tabular}{p{0.85\linewidth}r}
        \textcolor{NavyBlue}{#2} & \multicolumn{1}{m{3cm}}{\raggedleft \textsc{#1}}\\
        #3
    \end{tabular}
    \begin{tabular}{p{\linewidth}}
    \vspace{-0.3cm}
        \footnotesize{#4}
    \end{tabular}
    \vspace{-0.5cm}
}

\newcommand{\iproject}[3]{
    \begin{tabular}{p{0.85\linewidth}r}
        \textcolor{NavyBlue}{#2} & \multicolumn{1}{m{3cm}}{\raggedleft \textsc{#1}}\\
    \end{tabular}
    \begin{tabular}{p{\linewidth}}
    \vspace{-0.3cm}
        \footnotesize{#3}
    \end{tabular}
    \vspace{-0.5cm}
}

\newcommand{\projectlist}[2]{
    \begin{tabular}{p{\linewidth}}
        \textcolor{NavyBlue}{#1}\\
        \vspace{-0.3cm}
        \footnotesize{#2}
    \end{tabular}
    \vspace{-0.4cm}
}

\newcommand{\itemlist}[1]{
    \begin{tabular}{>{\raggedleft}p{1.5cm}p{0.9\linewidth}}
        #1
    \end{tabular}
}

\newcommand{\skill}[2]{
    \begin{tabular}{p{0.85\linewidth}r}
        #2 & \multicolumn{1}{m{3cm}}{\raggedleft \textsc{#1}}\\
    \end{tabular}
    \vspace{-0.5cm}
}

\newcommand{\github}{
    \includegraphics[height=12pt]{octocatvector.eps}
}

%----------------------------------------------------------------------------------------
% DOCUMENT BEGINS HERE
%----------------------------------------------------------------------------------------

\begin{document}

\font\fb=''[cmr10]'' % Change the font of the \LaTeX command under the skills section

\titleformat{\section}{\large\scshape\raggedright}{}{0em}{}[\titlerule] % Section formatting

% \pagestyle{scrheadings} % Print the headers and footers on all pages

% \addtolength{\voffset}{-0.5in} % Adjust the vertical offset - less whitespace at the top of the page
% \addtolength{\textheight}{3cm} % Adjust the text height - less whitespace at the bottom of the page
% \addtolength{\textwidth}{-2cm}

%----------------------------------------------------------------------------------------
% HEADER SECTION
%----------------------------------------------------------------------------------------

{
    \begin{tabular}{rl}
        \multirow{3}{0.75\linewidth}{
            {
                \addfontfeature{LetterSpace=20.0}\fontsize{24}{24}\selectfont\scshape
                Srijan R. Shetty
            }\\
            \hspace{0.15cm} \emph{Deptarment of Computer Science and Engineering}
        }

        & {\Large\Mobilefone} +(91) 9005900383 \\
        & {\Large\Info} \href{cse.iitk.ac.in/users/srijans}{cse.iitk.ac.in/users/srijans}\\
        & {\Large\Letter} \href{mailto:srijan.shetty@gmail.com}{srijan.shetty@gmail.com}\\
    \end{tabular}
}

%----------------------------------------------------------------------------------------
%   INTERESTS
%----------------------------------------------------------------------------------------
\section{Interests}

Programming Languages, Networks and Systems, Security, Web-Dev, NLP

%----------------------------------------------------------------------------------------
%	EDUCATION
%----------------------------------------------------------------------------------------
\section{Education}

\begin{tabular}{>{\raggedleft}p{1.5cm}p{15.5cm}r}

    \textsc{Current} & B. Tech in \textsc{Computer Science and Engineering} &   9.5/10.0\\
                     & \textbf{Indian Institute of Technology}, Kanpur\\

    \textsc{Jul 2011} & 12th Board, \textsc{CBSE} Board                    &   91.4\% \\
                       & \normalsize\textbf{The Emerald Heights International School}, Indore\\

    \textsc{Jul 2011} & 10th Board, \textsc{ICSE} Board                    &   95.4\% \\
                       & \normalsize\textbf{The Laurels School International}, Indore \\

\end{tabular}

%----------------------------------------------------------------------------------------
%	SCHOLARSHIPS AND CONFERENCES
%----------------------------------------------------------------------------------------
\section{Scholarships and Conferences}


\itemlist {
    \textsc{Jan 2011} & Kishore Vaigyanik Protsahan Yojana (KVPY) Fellowship \\
    \textsc{Dec 2008} & FIITJEE Talent Reward Examination Scholarship \\
    \textsc{Dec 2009} & CSIR Programme on Youth for Leadership in Science
                        held at Advanced Materials Research Institute, Bhopal\\
}


%----------------------------------------------------------------------------------------
%   SCHOLASTIC ACHIEVEMENTS
%----------------------------------------------------------------------------------------
\section{Scholastic Achievements}

\itemlist {
    \textsc{2011-2014}   & Recieved an \textbf{A* grade}, for exceptional performance in
                            \textbf{Computer, Internet and Network Security},
                            \textbf{Computer Networks},
                            \textbf{Computer Organization},
                            \textbf{Logic for Computer Science},
                            \textbf{Fundamentals of Computing},
                            \textbf{Introduction to Philosophy} \\
    \textsc{2012-2013}   & \textbf{Academic Excellence Award}, IIT Kanpur. \\
    \textsc{2011-2012}   & \textbf{Academic Excellence Award}, IIT Kanpur. \\
    \textsc{Monsoon 2012}   & \textbf{Academic Mentor}, \textbf{Fundamentals of Computer Science (ESC101)}. \\
                         & \footnotesize{Tutored a batch of students in ESC101 --- a core course on
                            basic concepts of programming --- the instructor-in-charge was \textbf{Professor
                            Sumit Ganguly}}\\
    \textsc{Jan 2010}    & Qualified for \textbf{Indian National Mathematics Olympaid} (INMO) \\
    \textsc{May 2011}    & Secured All India rank \textbf{1937} in Joint Entrance Exam (JEE) \\
    \textsc{Sep 2010}    & \textbf{AIR 17} in \textbf{Technothlon} (Techniche) conducted by Indian Institute of Technology Guwahati. \\
    \textsc{Jun 2010}   & Awarded \textbf{Top Scorer in English} by \textbf{Laurels School International} in ICSE 2009. \\
    \textsc{Dec 2009}    & Awarded \textbf{Overall Best Student} for displaying all round excellence. \\
    \textsc{Dec 2005}    & Awarded \textbf{Overall Best Student} for displaying all round excellence. \\
    \textsc{2000-2009} & Recipient of \textbf{Scholastic Excellence} awards for exemplary academic
                           performance throughout 2nd to 12th grade. \\
    \textsc{---} & Secured AIR \textbf{140 in 8th, 154 in 9th and 429 in 11th} National Cyber Olympaid.\\
    \textsc{---} & Secured AIR \textbf{54 in 9th} National Science Olympaid.\\
}


%----------------------------------------------------------------------------------------
%   SKILL SET
%----------------------------------------------------------------------------------------
\section{Skill Set}

\itemlist {
    \textsc{Languages} %
            & \textbf{JavaScript} (Expert), \textbf{Python} (Proficient), \textbf{C++}
              (Proficient), \textbf{C\#} (Efficient)\\
    \textsc{Web Dev} %
            & \textbf{HTML5}, \textbf{CSS}, \textbf{SQL and NoSQL}, \textbf{AngularJS}, \textbf{Nodejs}\\
    \textsc{Tools} %
            & \textbf{Git} (Expert), \textbf{Shell Scripting} (Proficient),
              \textbf{Prezi} (Proficient), \textbf{Vim} (Expert), \textbf{LaTeX}
              (Proficient)\\
}

%----------------------------------------------------------------------------------------
%   WORK EXPERIENCE
%----------------------------------------------------------------------------------------

\section{Work Experience}

\work {May-Jul 2014}
      {Research Intern at \textsc{Microsoft Research India}, Bangalore}
      {
         \begin{itemize}[leftmargin=*]
             \item Analysed the implementations of different distributed systems and
                 databases architectures like \textbf{Google MapReduce}, \textbf{Google FileSystem},
                 \textbf{Google MillWheel} \textbf{Yahoo! PNUTS}, \textbf{Amazon DynamoDB}, \textbf{Apache
                 Kafka}, \textbf{Apache Spark}, \textbf{Apache Hadoop}, \textbf{Apache Storm},
                 \textbf{Microsoft Dryad}, and \textbf{Microsoft DryadLINQ}
             \item CScale provides a declarative distributed systems programming
                 model using LINQ. A programmer with no distributed system
                 experience can specify the computation intent in LINQ and
                 CScale handles scaling the system and fault tolerance.
             \item Optimized \textbf{CScale for concurrency} using \textbf{TPL
                     Dataflow}.
             \item Used \textbf{TPL Dataflow} and \textbf{Lightweight Tasks} to
                 implement concurrent pipeline processing of messages.
             \item Designed and implemented a dataflow pipeline for processing
                 partially ordered messages in order to create a new totally
                 ordered sequence.
             \item Performed \textbf{performance testing} of the concurrency optimizations
                 in real world scenarios leading a gain in processing time.
             \item Performed \textbf{failure testing} of the Programming model
                 to uncover detrimental replication errors.
         \end{itemize}
     }

\work {May-Jul 2013}
      {Software Development at \textsc{Aurus Networks'}, Bangalore}
      {
          \begin{itemize}[leftmargin=*]
              \item Worked on \textbf{Aurus Networks} private MOOC offering \textbf{CourseHub}.
              \item Ported the legacy flash based web player to a technology
                  agnostic \textbf{HTML5/JavaScript} web player leading to a
                  uniform cross-platform viewing experience.
              \item Developed a \textbf{XMPP} based \textbf{real-time chat service} using the existing
                  Jabber framework, for conversations during live lectures.
          \end{itemize}
      }

%----------------------------------------------------------------------------------------
%  COURSE PROJECTS
%----------------------------------------------------------------------------------------
\section{Course Projects}

\lproject {Monsoon 2014}
          {Multi Factor Authentication in OpenVPN}
          {\textsc{\raggedright Mozilla Winter of Security}, Guillaume Destuynder and Professor Dheeraj Sanghi}
          {
              \begin{itemize}[leftmargin=0.5cm]
                  \item One of the eleven projects selected as a part of Mozilla's Winter of Security Initiative 2014.
                  \item The objective of the project is to implement true arbitary multi factor authentication support in
                      OpenVPN; and to implement session-support for the second factor.
                  \item \href{https://wiki.mozilla.org/Security/Mentorships/MWoS/2014/OpenVPN\_MFA}%
                      {https://wiki.mozilla.org/Security/Mentorships/MWoS/2014/OpenVPN\_MFA}
              \end{itemize}
          }

\lproject {Winter 2014}
          {JavaScript to MIPS Compiler}
          {\textsc{\raggedright Compilers}, Professor Subhajit Roy}
          {
              \begin{itemize}[leftmargin=0.5cm]
                  \item Implemented an end-to-end compiler to compile ECMAScript5.1 to machine code for MIPS, in Python.
                  \item Abstracted the compiler into modules corresponding to lexing, parsing, three address code generation
                      and machine code generation, with swappable modules.
                  \item Designed a standard library to handle printing of different data types.
                  \item Implemented functional programming concepts and recursion.
                  \item Circumvented runtime support by implementing static types and type annotations.
                  \item \href{https://github.com/srijanshetty/javascript-compiler} {https://github.com/srijanshetty/javascript-compiler}
              \end{itemize}
          }

\lproject {Winter 2014}
          {Hindi author attribution}
          {\textsc{\raggedright Artificial Intelligence}, Professor Amitabha Mukherjee}
          {
             \begin{itemize}[leftmargin=0.5cm]
                 \item Clustered multiple authors using K-means unsupervised clustering.
                 \item Classified each author using Support Vector Machine Classification using Radial Basis Kernel function.
                 \item Used unigrams, bigrams and trigrams as features for clustering and Multiple Discrimanat Analysis for comparitive analysis.
                 \item The inherent difficult in the project was the unavailability of stemmers and POS taggers for Hindi.
                 \item \href{https://github.com/srijanshetty/author-attribution}{https://github.com/srijanshetty/author-attribution}
             \end{itemize}
         }

\lproject {Monsoon 2013}
          {NachOS}
          {\textsc{Operating Systems}, Professor Mainak Chaudhari}
          {
             \begin{itemize}[leftmargin=0.5cm]
                 \item Implemented standard system calls in NachOS.
                     \href{https://github.com/srijanshetty/nachos-syscalls/}{https://github.com/srijanshetty/nachos-syscalls/}
                 \item Assessed the performance different Process scheduling algorithms: UNIX Scheduling, First in First Out,
                     Round Robin, Shortest Job First and Non-preemptive job scheduling.
                     \href{https://github.com/srijanshetty/nachos-scheduling}{https://github.com/srijanshetty/nachos-scheduling}
                 \item Evaluated the performance of different page replacement algorithms: Random Page Allocation, First in First Out,
                     Least Recently Used(LRU) and LRU Clock.
                     \href{https://github.com/srijanshetty/nachos-final/}{https://github.com/srijanshetty/nachos-final/}
              \end{itemize}
          }

\lproject {Monsoon 2013}
          {P2P File Sharing and Streaming}
          {\textsc{Computer Networks}, Professor Dheeraj Sanghi}
          {
             \begin{itemize}[leftmargin=0.5cm]
                 \item Concieved a protocol for peer-to-peer based file and media transfer.
                 \item Implemented the designed protocol to create a CLI agent for the same.
                 \item Leveraged Node.js to handle high number of concurrent connections.
                  \item \href{https://github.com/srijanshetty/nodesock}{https://github.com/srijanshetty/nodesock}
             \end{itemize}
          }

\lproject {Monsoon 2013}
          {IP Spoofing}
          {\textsc{Computer Networks}, Professor Dheeraj Sanghi}
          {
              \begin{itemize}[leftmargin=0.5cm]
                  \item Generated raw Internet Protocol (IP) packets with spoofed source address.
                  \item Exploited spoofed packets to perform Smurfing, ARP Poisoning and SYN Flooding.
                  \item Tested the implementation in a secure subnet.
              \end{itemize}
          }

\lproject {Monsoon 2013}
          {Prolog Shortest Paths and Erlang Traffic Server}
          {\textsc{Principles of Programming Languages}, Professor Piyush Kurur}
          {
               \begin{itemize}[leftmargin=0.5cm]
                   \item Implemented shortest path algorithm in Prolog using Logic Programming on
                       the Delhi Metro route.
                   \item Implemented a simple traggice server in Erlang using message passing concurrency and
                       tail call recursion.
               \end{itemize}
           }

\lproject {Winter 2013}
          {8-bit General Purpose Computer on a FPGA}
          {\textsc{Computer Organization}, Professor Subhajit Roy}
          {
              \begin{itemize}[leftmargin=0.5cm]
                  \item Designed an Instruction Set Architecture (ISA) for a 8-bit General Purpose
                      Computer with a load-store architecture.
                  \item Programmed a \textbf{Xilinx Spartan 3 FPGA} in System Verilog to implement
                      the ISA.
                  \item Encoded a simple assembly language to compile to the machine code.
                  \item Accomplished recursion, jumps and loops.
                  \item \href{https://github.com/srijanshetty/220\_y11} {https://github.com/srijanshetty/220\_y11}
              \end{itemize}
          }

\lproject {Winter 2013}
          {Conference Dates Web Crawler}
          {\textsc{Computing Laboratory}}
          {
              \begin{itemize}[leftmargin=0.5cm]
                   \item Developed a python based web crawler to crawl paper submission deadlines for a given paper
                       and conference name.
                   \item Quantitatively tested the advantage of Depth First Search over Bread First Search for crawling.
                   \item Utilized regular expressions, BeautifulSoup and urllib to perform crawling.
                   \href{https://github.com/srijanshetty/crawler} {https://github.com/srijanshetty/crawler}
              \end{itemize}

          }

\lproject {Monsoon 2011}
          {No situation is unique and certain moral principles can be applied across all situation}
          {\textsc{Introduction to Philosophy}, Professor Vineet Sahu}
          {
               \begin{itemize}[leftmargin=0.5cm]
                   \item Illustrated the existence of a fundamental similarity of different situtations, by
                       a gross simplification of different situations.
                   \item Justified the use certain moral principles across all situations by leveraging the
                       above stated hypothesis.
               \end{itemize}
           }

%----------------------------------------------------------------------------------------
%  INDEPENDENT PROJECTS
%----------------------------------------------------------------------------------------
\section{Independent Projects}

\iproject {Jul 2014}
          {OARS}
          {
               \begin{itemize}[leftmargin=0.5cm]
                   \item Scraped the institute academic records to create a database of all courses offered by the
                       institute.
                   \item Created an AngularJS based frontend search for the scraped data with a focus on ease of use
                       and accessibility.
                   \item \href{https://navya.github.io/oars}{https://navya.github.io/oars}
               \end{itemize}
           }

\iproject {Mar 2014}
          {Get Your Personal Homepage (GYPH)}
          {
               \begin{itemize}[leftmargin=0.5cm]
                   \item Eased the process of creating minimalistic websites for not-so-tech-savvy students.
                   \item Provided a directly editable interface for editing websites using JavaScript and HTML5.
                   \item Enabled users to use custom themes and download the created website for personal use.
                   \item \href{http://gyph2.herokuapp.com/} {http://gyph2.herokuapp.com/}
               \end{itemize}
           }

\iproject {Aug 2013}
          {ShuffleRun}
          {
              \begin{itemize}[leftmargin=0.5cm]
                  \item Designed a web app which selects a music track from a user's music library based on his current running speed.
                  \item Pitched the idea at \textbf{Yahoo! HackU 2013}.
                  \item Recieved an honorable mention in \textbf{Yahoo!  HackU 2013} for the created hack.
                  \item \href{https://github.com/srijanshetty/ShuffleRun} {https://github.com/srijanshetty/ShuffleRun}
              \end{itemize}
          }

\lproject {Summer 2012}
          {Voicing, Archiving and Networking the Information \textsc{(VANI)}, IIT Kanpur}
          {Professor Manindra Agarwal, Professor T. V. Prabhakar}
          {
              \begin{itemize}[leftmargin=0.5cm]
                  \item Conceptualized the idea of VANI - an extensible student community platform consisting
                      of a Student Wiki, Forums, Community Search, Calendar and Lost and Found.
                  \item Developed the Lost and Found module using \textbf{Drupal} for VANI.
                  \item Conducted sessions on the extending VANI to create rich applications.
              \end{itemize}
          }

\iproject {Summer 2012}
          {GNU/Linux Exploration, Programming Club, IIT Kanpur}
          {
              \begin{itemize}[leftmargin=0.5cm]
                  \item Explored various facets of \textbf{GNU/Linux} like the file system, process management,
                      memory management, shell interface etc.
                  \item Documented all salient points for use by the freshmen of the insititue.
                  \item \href{https://docs.google.com/document/d/1ZHO9w36aoq3oaZBR4Um1AOmDfiTDAEgM6baQAu3icw4/edit?usp=sharing}{https://docs.google.com/document/d/1ZHO9w36aoq3oaZBR4Um1AOmDfiTDAEgM6baQAu3icw4/edit?usp=sharing}
              \end{itemize}
          }

\projectlist {Projects on Github}
             {
                 \begin{itemize}[leftmargin=0.5cm]
                     \item \textbf{Dotfiles}: an opinionated work flow on Linux Systems.
                         \href{https://github.com/srijanshetty/dotfiles} {https://github.com/srijanshetty/dotfiles}
                     \item \textbf{Prezto}: a fork of the Prezto ZSH framework
                         \href{https://github.com/srijanshetty/prezto} {https://github.com/srijanshetty/prezto}
                     \item \textbf{oh-my-zsh}: a fork of the oh-my-zsh ZSH framework
                         \href{https://github.com/srijanshetty/oh-my-zsh} {https://github.com/srijanshetty/oh-my-zsh}
                     \item \textbf{DS}: implementation of certain algorithms and data structures
                         \href{https://github.com/srijanshetty/DS} {https://github.com/srijanshetty/DS}
                     \item \textbf{Custom}: handy configurations and shortcuts for ZSH
                         \href{https://github.com/srijanshetty/custom} {https://github.com/srijanshetty/custom}
                     \item \textbf{Notes}: notes on various computing concepts
                         \href{https://github.com/srijanshetty/notes} {https://github.com/srijanshetty/notes}
                 \end{itemize}
             }

\projectlist {Web Development}
             {
                 \begin{itemize}[leftmargin=0.5cm]
                     \item \textbf{Udghosh}: The annual sports-fest of IIT Kanpur.
                         \href{https://github.com/srijanshetty/udghosh-final}{https://github.com/srijanshetty/udghosh-final}
                     \item \textbf{Hall 5}: Fifth Hall of Residence, IIT Kanpur.
                         \href{http://www.iitk.ac.in/hall5} {http://www.iitk.ac.in/hall5}
                     \item \textbf{ALI}: Antaragni Leadership Initiative.
                         \href{www.antargni.in/ali} {www.antargni.in/ali}
                 \end{itemize}
             }

%----------------------------------------------------------------------------------------
%   OTHER SKILLS
%----------------------------------------------------------------------------------------

\section {Social, Leadership and Artistic Skills}

\projectlist {Hacker, Navya, FOSS Group IIT Kanpur}
             {
                   \begin{itemize}[leftmargin=0.5cm]
                       \item Navya is the resident Free and Open Source Software Group of IIT Kanpur.
                       \item Promoted the FOSS initiative in campus by organizing student lectures and meet-ups.
                       \item Created student-centric applications like Course Search and Student Search, to circumvent
                           their legacy institute counterparts.
                       \item \href{https://github.com/navya} {https://github.com/navya}
                   \end{itemize}
             }

\iproject {2014}
          {Head, Major Events and Competitions, Antaragni'14}
          {
               \begin{itemize}[leftmargin=0.5cm]
                   \item Organized Mr. and Ms. Fresher's competition for the first time in IIT Kanpur.
                   \item Worked alongside the Creative team to promote Antaragni through our Social Media Campaign
                       and Social Campaign which focussed on road safety.
               \end{itemize}
          }

\iproject {2013}
          {Hospitality Coordinator, Antaragni'13}
          {
               \begin{itemize}[leftmargin=0.5cm]
                   \item Transformed the methodology of inviting colleges all around the country by focussing on
                       calling college societies rather than the cultural union.
                   \item Worked with a team of five fellow coordinators in planning the accommodation of 1500 participating students
                       from all over the country.
                   \item Chalked out the logistics for the hospitality of \textbf{1500 students} who
                       participated in festival.
                   \item Chalked out the logistics of the hospitality and accomodation of \textbf{1500 students} who
                       participated in the festival with my fellow coordinators.
                   \item Helmed a team of 40 secretaries and 80 volunteers tasked with ensuring a flawless conduction of the
                       festival.
                   \item Handled security during the four days of the Festival from 24$^{th}$ to 27$^{th}$ October.
               \end{itemize}
          }

\iproject {2013 -- 2014}
          {Community Service, English Proficiency Programme}
          {
               \begin{itemize}[leftmargin=0.5cm]
                   \item A brainchild of \textbf{Professor Bhaskardas Gupta}, English Proficiency
                       Programme tries to impart a functional knowledge of English to students.
                   \item Worked as a English tutor in the pilot programme which was aimed at helping
                       under-privileged students from and around IIT Kanpur campus.
                   \item Helped in organizing the second phase of the programme, aimed at school teachers
                       by meetings principals from schools in and around IIT Kanpur.
               \end{itemize}
          }

\iproject {2013 -- 2014}
          {Member, Gymkhana Review Committee}
          {
               \begin{itemize}[leftmargin=0.5cm]
                   \item The GRC was established to revamp the Students' Gymkhana of IIT Kanpur,
                       which is responsible for all the student activities of the institute.
                   \item Chaired the meetings on \textbf{Extended Orientation of Incoming Freshmen}.
                   \item Contributed actively as a member of academic, senate and activities sub-committees of the GRC.
               \end{itemize}
          }

\skill {2013 -- pre}
       {Active blogger at \href{srijanshetty.quora.com} {srijanshetty.quora.com}}

\skill {2013 -- 2014}
       {Editor-in-chief, Vox Populi, the campus newsletter.}

\skill {2012 -- 2013}
       {Senator, Students' Senate, IITK Y11 batch.}

\skill {2012 -- 2013}
       {Secretary, English Literary Society}

\skill {2010 -- 2011}
       {Basic training in \textbf{Indian Classical Music}. (Prayag Sangeet Samiti Allahabad).}

\skill {2009}
       {Qualified for the semi-finals of \textbf{Outlook SpeakOut Debate}.}

%----------------------------------------------------------------------------------------
%   COURSE WORK
%----------------------------------------------------------------------------------------

\section{CourseWork}

\begin{center}
    \begin{tabular}{>{\raggedleft}p{8cm}|p{8cm}}

        Theory of Computation (CS340) & (CS201) Discreet Mathematics \\
        Abstract Algebra (CS203A) & (CS202A) Logic for Computer Science \\
        Data Structure and Algorithms (CS210) & (CS345) Design and Analysis of Algorithms \\
        Operating Systems (CS330) & (CS335) Compilers \\
        Computer Networks (CS425) &  (CS628) Computer and Internet Security \\
        Computer Organization (CS220) & (CS350) Principles of Programming Languages \\
        Artificial Intelligence (CS365) & (CS251 \& CS252) Computing Laboratory \\
        Computational Methods in Engineering (ESO208A) & (CS300) Technical Communication \\
        Linear Algebra (MTH102) &  (MTH101) Multivariate Calculus \\
    \end{tabular}
\end{center}

\end{document}
