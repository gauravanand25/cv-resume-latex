%----------------------------------------------------------------------------------------
%	PACKAGES AND OTHER DOCUMENT CONFIGURATIONS
%----------------------------------------------------------------------------------------

\documentclass[a4paper,10pt]{article} % Default font size and paper size

\usepackage{fontspec} % For loading fonts
\defaultfontfeatures{Mapping=tex-text}
\setmainfont[SmallCapsFont = Fontin SmallCaps]{Fontin} % Main document font

\usepackage{xunicode,xltxtra,url,parskip} % Formatting packages

\usepackage{marvosym} % Allows the use of symbols

\usepackage{scrpage2} % Provides headers and footers configuration

\usepackage[usenames,dvipsnames]{xcolor} % Required for specifying custom colors

% \usepackage[big]{layaureo} % Margin formatting of the A4 page, an alternative to layaureo can be \usepackage{fullpage}
% To reduce the height of the top margin uncomment: \addtolength{\voffset}{-1.3cm}

\usepackage{hyperref} % Required for adding links	and customizing them
\definecolor{linkcolour}{rgb}{0,0.2,0.6} % Link color
\hypersetup{colorlinks,breaklinks,urlcolor=linkcolour,linkcolor=linkcolour} % Set link colors throughout the document

\usepackage{tabularx,colortbl} % Advanced table configurations

\usepackage{titlesec} % Used to customize the \section command
\titleformat{\section}{\Large\scshape\raggedright}{}{0em}{}[\titlerule] % Text formatting of sections

\titlespacing{\section}{0pt}{3pt}{3pt} % Spacing around sections

\usepackage{enumitem} % For left margin removal in itemize

\usepackage{array} % Custom arrangement for the columns

%----------------------------------------------------------------------------------------
% PAGE GEOMETRY
%----------------------------------------------------------------------------------------

\usepackage[margin=2cm, lmargin=0.5cm, rmargin=0.5cm]{geometry} % The page Geometry

%----------------------------------------------------------------------------------------
% CUSTOM ENVIRONMENTS
%----------------------------------------------------------------------------------------
\newcommand{\project}[3]{
    \begin{tabular}{>{\raggedleft}p{2.2cm}|p{17cm}}
        \textsc{#1} & #2 \\
                    & \footnotesize{#3} \\
    \end{tabular}
}

\newcommand{\projectlist}[3]{
    \begin{tabular}{>{\raggedleft}p{2.2cm}|p{17cm}}
        \textsc{#1} & #2
                    \footnotesize{#3}
    \end{tabular}
}

\newcommand{\itemlist}[1]{
    \begin{tabular}{>{\raggedleft}p{2.2cm}p{17cm}}
        #1
    \end{tabular}
}

%----------------------------------------------------------------------------------------
% DOCUMENT BEGINS HERE
%----------------------------------------------------------------------------------------

\begin{document}

\font\fb=''[cmr10]'' % Change the font of the \LaTeX command under the skills section

\titleformat{\section}{\large\scshape\raggedright}{}{0em}{}[\titlerule] % Section formatting

\pagestyle{scrheadings} % Print the headers and footers on all pages

% \addtolength{\voffset}{-0.5in} % Adjust the vertical offset - less whitespace at the top of the page
% \addtolength{\textheight}{3cm} % Adjust the text height - less whitespace at the bottom of the page
% \addtolength{\textwidth}{-2cm}

%----------------------------------------------------------------------------------------
% FOOTER SECTION
%----------------------------------------------------------------------------------------

\renewcommand{\headfont}{\normalfont\rmfamily\scshape} % Font settings for footer

\cofoot{
    \addfontfeature{LetterSpace=20.0}\fontsize{10.5}{17}\selectfont % Letter spacing and font size

    I-308, IIT Kanpur {\large\textperiodcentered} Kanpur {\large\textperiodcentered} UP 208016\\ % Your mailing address
    {\Large\Letter} srijan.shetty@gmail.com \ {\Large\Telefon} +(91) 9005900383 % Your email address and phone number
}

%----------------------------------------------------------------------------------------
% HEADER SECTION
%----------------------------------------------------------------------------------------

{
    {\addfontfeature{LetterSpace=20.0}\fontsize{24}{24}\selectfont\scshape
    Srijan R. Shetty}\\
}

%----------------------------------------------------------------------------------------
%   INTERESTS
%----------------------------------------------------------------------------------------
\section{Interests}

Programming Languages, Distributed Systems, Systems and Security, Web-Dev, NLP

%----------------------------------------------------------------------------------------
%	EDUCATION
%----------------------------------------------------------------------------------------
\section{Education}

\begin{tabular}{>{\raggedleft}p{2.2cm}p{14cm}r}

    \textsc{Current} & B. Tech in \textsc{Computer Science and Engineering} &   9.6/10.0\\
                     & \textbf{Indian Institute of Technology}, Kanpur\\

    \textsc{July 2011} & 12th Board, \textsc{CBSE} Board                    &   91.4\% \\
                       & \normalsize\textbf{The Emerald Heights International School}, Indore\\

    \textsc{July 2011} & 10th Board, \textsc{ICSE} Board                    &   95.4\% \\
                       & \normalsize\textbf{The Laurels School International}, Indore \\

\end{tabular}

%----------------------------------------------------------------------------------------
%	SCHOLARSHIPS AND CONFERENCES
%----------------------------------------------------------------------------------------
\section{Scholarships and Conferences}


\itemlist {
    \textsc{Jan 2011} & Kishore Vaigyanik Protsahan Yojana (KVPY) Fellowship \\
    \textsc{Dec 2008} & FIITJEE Talent Reward Examination Scholarship \\
    \textsc{Dec 2009} & CSIR Programme on Youth for Leadership in Science (CPYLS camp),
                        held at Advanced Materials Research Institute, Bhopal\\
}


%----------------------------------------------------------------------------------------
%   SCHOLASTIC ACHIEVEMENTS
%----------------------------------------------------------------------------------------
\section{Scholastic Achievements}

\itemlist {
    \textsc{2011-2014}   & Recieved an \textbf{A* grade}, for exceptional performance in
                            \textbf{Computer, Internet and Network Security},
                            \textbf{Computer Networks},
                            \textbf{Computer Organization},
                            \textbf{Logic for Computer Science},
                            \textbf{Fundamentals of Computing},
                            \textbf{Introduction to Philosophy} \\
    \textsc{2012-2013}   & \textbf{Academic Excellence Award}, IIT Kanpur. \\
    \textsc{2011-2012}   & \textbf{Academic Excellence Award}, IIT Kanpur. \\
    \textsc{Fall 2012}   & \textbf{Academic Mentor}, \textbf{Fundamentals of Computer Science (ESC101)}. \\
                         & \footnotesize{Tutored a batch of students in ESC101 --- a core course on
                            basic concepts of programming --- the instructor-in-charge was \textbf{Professor
                            Sumit Ganguly}}\\
    \textsc{Jan 2010}    & Qualified for \textbf{Indian National Mathematics Olympaid} (INMO) \\
    \textsc{May 2011}    & Secured All India rank \textbf{1937} in Joint Entrance Exam (JEE) \\
    \textsc{Sep 2010}    & \textbf{AIR 17} in \textbf{Technothlon} (Techniche) conducted by Indian Institute of Technology Guwahati. \\
    \textsc{June 2010}   & Awarded \textbf{Top Scorer in English} by \textbf{Laurels School International} in ICSE 2009. \\
    \textsc{Dec 2009}    & Awarded \textbf{Overall Best Student} for displaying all round excellence. \\
    \textsc{Dec 2005}    & Awarded \textbf{Overall Best Student} for displaying all round excellence. \\
    \textsc{2000 -- 2009} & Recipient of \textbf{Scholastic Excellence} awards for exemplary academic
                           performance throughout 2nd to 12th grade. \\
    \textsc{---} & Secured AIR \textbf{140 in 8th, 154 in 9th and 429 in 11th} National Cyber Olympaid.\\
    \textsc{---} & Secured AIR \textbf{54 in 9th} National Science Olympaid.\\
}


%----------------------------------------------------------------------------------------
%   SKILL SET
%----------------------------------------------------------------------------------------
\section{Skill Set}

\itemlist {
    \textsc{Languages} %
            & \textbf{C\#}, \textbf{C++},\textbf{Python}, \textbf{JavaScript} \\
    \textsc{Web Dev} %
            & \textbf{HTML5}, \textbf{CSS}, \textbf{JavaScript}, \textbf{Flash}, \textbf{SQL},
              \textbf{AngularJS}, \textbf{Nodejs}, \textbf{NoSQL}\\
    \textsc{Tools} %
            & \textbf{Git}, \textbf{Shell Scripting}, \textbf{Prezi}, \textbf{Vim}, \textbf{LaTeX}\\
}

%----------------------------------------------------------------------------------------
%   WORK EXPERIENCE
%----------------------------------------------------------------------------------------

\section{Work Experience}

\projectlist {May-Jul 2014}
         {Research Intern at \textsc{Microsoft Research India}, Bangalore}
         {
            \begin{itemize}[leftmargin=*]
                \item Optimized \textbf{CScale for concurrency} using \textbf{TPL Dataflow}
                    for dataflow computations and pipeline processing.
                \item CScale provides a declarative distributed systems programming
                    model using LINQ; it handles scaling and failures for the programmer.
                \item Performed \textbf{failure testing and performance testing}
                    of the concurrency optimizations in real world scenarios.
            \end{itemize}
        }

\projectlist {May-Jul 2013}
         {Software Development at \textsc{Aurus Networks'}, Bangalore}
         {
            \begin{itemize}[leftmargin=*]
                \item Worked on \textbf{Aurus Networks} MOOC offereing \textbf{CourseHub}.
                \item Using \textbf{HTML5, javascript, CSS3}, I ported the legacy
                      \textbf{Flash} based online media player to a multi-platform
                      Flash/HTML5 media player.
                  \item Developed a \textbf{XMPP} based \textbf{real-time chat service} for
                      conversations during live lectures.
            \end{itemize}
        }

%----------------------------------------------------------------------------------------
%  PROJECTS
%----------------------------------------------------------------------------------------
\section{Projects}

\project {Fall 2014}
         {Multi Factor Authentication in OpenVPN, \textsc{\raggedright Mozilla Winter of Security}}
         {This project was one of the eleven projects selected for Mozilla Winter of Security's 2014 edition.
         Under the guidance of \textbf{Guillaume Destuynder} and \textbf{Professor Dheeraj Sanghi},
         the project aims to add true multi factor authentication support to OpenVPN.}

\project {Spring 2014}
         {JavaScript to MIPS Compiler, \textsc{\raggedright Compilers}}
         {An end-to-end compiler for a subset of ECMAScript5.1 which compiled to MIPS was implemented in Python.
          No runtime support was provided ergo, static types and type annotations were used instead.
          The project was undertaken in the aegis of \textbf{Professor Subhajit Roy}.
          \href{https://github.com/srijanshetty/javascript-compiler} {[Git Repo]}}

\project {Spring 2014}
         {Hindi author attribution, \textsc{\raggedright Artificial Intelligence}}
         {The first half of this project was a multi-class clustering problem,
          wherein authors were to be clustered according to their writing style.
          The features used for the same were, word frequency counts and collocation
          frequencies.  The second half of the project was to classify the text snippets
          into different authors using supervised learning techniques.
          The project was supervised by \textbf{Professor Amitabha Mukherjee} and was
          inherently difficult due to the unavailability of stemmers and POS taggers for Hindi.
          \href{https://github.com/srijanshetty/author-attribution} {[Git Repo]}}

\project {Fall 2013}
         {NachOS, \textsc{Operating Systems}}
         {The coursework included implementation of system calls; process
          management and scheduling; memory management schemes and page replacement
          algorithms in \textbf{NachOS}. \textbf{Professor Mainak Chaudhuri}, was
          the professor-in-charge}\\

\project {Fall 2013}
         {P2P File Sharing and Streaming, \textsc{Computer Networks}}
         {A Node.js based local file server and media playback server was implemented
          to understand the underpinnings of Transport Layer Protocols, socket programming,
          and P2P architecture.  The project was supervised by \textbf{Professor Dheeraj Sanghi}.
          \href{https://github.com/srijanshetty/nodesock} {[Git Repo]}}

\project {Fall 2013}
         {IP Spoofing, \textsc{Computer Networks}}
         {Under the guidance of \textbf{Professor Dheeraj Sanghi}, different methodologies
          of spoofing IP addresses and different uses of spoofing IPs were implemented and tested
          in a secure environment.}

\project {Fall 2013}
         {Metro Travel Plan(Prolog), \textsc{Principles of Programming Languages}}
         {Implementation of shortest paths in Prolog on the Delhi Metro route.
          The professor-in-charge was \textbf{Professor P. Kurur}}

\project {Fall 2013}
         {Erlang Traffic Server, \textsc{Principles of Programming Languages}}
         {A simple traffic server implemented in Erlang using message passing
          concurrency and tail call recursion functional aspects of the language under the
          guidance of \textbf{Professor P. Kurur}}

\project {Spring 2013}
         {8-bit General Purpose Computer on a FPGA, \textsc{Computer Organization}}
         {The GPC had a load-store architecture and was written in System Verilog and
          implemented on a \textbf{Xilinx Spartan 3 FPGA}.  The project was under the
          supervision of \textbf{Professor Subhajit Roy}.  It had a limited, but powerful
          Instruction Set Architecture which could implement recursive functions, jumps,
          loops  other basic building blocks of a simple assembly language.
          \href{https://github.com/srijanshetty/220_y11} {[Git Repo]}}

\project {Spring 2013}
         {Conference Dates Web Crawler, \textsc{Computing Laboratory}}
         {A python based web crawler which could obtain paper submission
          deadlines of a provided conference name.  BeautifulSoup and urllib along
          with simple pattern matching through regular expressions were used to implement crawling.
          \href{https://github.com/srijanshetty/crawler} {[Git Repo]}}

\project {Fall 2011}
         {"No situation is unique and certain moral principles can be applied across all situation",
          \textsc{Introduction to Philosophy}}
         {An argumentative discourse proving that certain moral principles
          can be applied throughout; by proving different situations have a certain
          fundamental similarity.  Albeit this would require a gross simplification of
          different situations.  The reviewing instructor was \textbf{Professor Vineet Sahu}}

%----------------------------------------------------------------------------------------
%  SIDE PROJECTS
%----------------------------------------------------------------------------------------
\section{Side Projects}

\project {July 2014}
         {OARS}
         {OARS is a \textbf{AngularJS} based front-end for the
          courses offered in IIT Kanpur. It was created as a
          hack to alleviate the pain of the students in using the
          sluggish legacy course search. The data was obtained by
          scraping the official data portal of IIT Kanpur, and the
          project is now maintained by \textbf{Navya}
          \href{https://github.com/navya/oars} {[Git Repo]} }

\project {March 2014}
         {GYPH}
         {GYPH or Get Your Personal Homepage
          is a simple hassle-free homepage designer for students who are not
          proficient in web technologies. It provides a WYSIWIG
          interface for editing themes and downloading them for personal use.
          \href{http://gyph2.herokuapp.com/} {[Website]} }

\project {August 2013}
         {ShuffleRun}
         {ShuffleRun is a web based music player which selects a track from the
          users library according to her running speed.
          The project recieved an honorable mention in \textbf{Yahoo!  HackU 2013} IIT Kanpur round.
          \href{https://github.com/srijanshetty/ShuffleRun} {[Git Repo]}}

\project {July 2013}
         {\href{www.udghosh.org}{Website of Udghosh}, the annual sports-fest of IIT Kanpur}

\project {Dec 2012}
         {\href{http://www.iitk.ac.in/hall5}{Website of Hall 5}, IIT Kanpur }

\project {Summer 2012}
         {Voicing, Archiving and Networking the Information \textsc{(VANI)}, IIT Kanpur}
         {Vani is an \textbf{extensible} platform
          made for the campus community consisting of Wiki, Forums,
          Community search, Calendars etc.; built using \textbf{Drupal}
          under the mentorship of \textbf{Professor Manindra Agarwal} and
          \textbf{Professor T. V. Prabhakar} }

\project {Summer 2012}
         {GNU/Linux Exploration, Programming Club, IIT Kanpur}
         {The project involved exploration of various facets of
          \textbf{GNU/Linux} like the file system, process management,
          memory management, shell interface etc. And was
          undertaken under the aegis of \textbf{Programming Club, IIT Kanpur}
          \href{https://docs.google.com/document/d/1ZHO9w36aoq3oaZBR4Um1AOmDfiTDAEgM6baQAu3icw4/edit?usp=sharing}
          {[Documentation]} }

\projectlist {July 2013}
         {Projects on Github}
         {
             \begin{itemize}[leftmargin=*]
                 \item \textbf{Dotfiles}: an opinionated work flow on Linux Systems.
                     \href{https://github.com/srijanshetty/dotfiles} {[Git Repo]}
                 \item \textbf{Prezto}: a fork of the Prezto ZSH framework
                     \href{https://github.com/srijanshetty/prezto} {[Git Repo]}
                 \item \textbf{oh-my-zsh}: a fork of the oh-my-zsh ZSH framework
                     \href{https://github.com/srijanshetty/oh-my-zsh} {[Git Repo]}
                 \item \textbf{DS}: implementation of certain algorithms and data structures
                     \href{https://github.com/srijanshetty/DS} {[Git Repo]}
             \end{itemize}
         }

%----------------------------------------------------------------------------------------
%   OTHER SKILLS
%----------------------------------------------------------------------------------------

\section {Social, Leadership and Artistic Skills}

\begin{tabular}{>{\raggedleft}p{2.2cm}p{15cm}}

    ---   & Active blogger at \href{srijanshetty.quora.com} {srijanshetty.quora.com} \\

    2013-2014    & \textbf{Hacker, Navya, FOSS Group IIT Kanpur}
                   \footnotesize{
                       \begin{itemize}[leftmargin=*]
                           \item Navya is the resident Free and Open Source Software Group of IIT Kanpur.
                           \item Promoted the FOSS initiative by giving student lectures.
                               And worked on student-centred projects like Course Search and Student Search.
                               \href{https://github.com/navya} {[Git Repo]}
                       \end{itemize}
                   }\\

                   2013-2014    & \textbf{Volunteer, English Proficiency Programme (Community Service)}
                   \footnotesize{
                       \begin{itemize}[leftmargin=*]
                           \item A brainchild of \textbf{Professor Bhaskardas Gupta}, English Proficiency
                               Programme tries to impart a functional knowledge of English to students.
                           \item I assisted in the first leg of the programme which focussed in helping
                               under-privileged students learn English.
                       \end{itemize}
                   }\\

    2014 & \textbf{Head, Major Events and Competitions, Antaragni'14}
           \footnotesize{
               \begin{itemize}[leftmargin=*]
                   \item Oversaw a team of \textbf{150} students for a span of 8 months.
                   \item Chalked out the logistics for the hospitality and accommodation of \textbf{1500 students} who
                       participated in festival.
                   \item Made sure everything went smoothly during the four days of the Festival from 24$^{th}$
                       to 27$^{th}$ October.
               \end{itemize}
           }\\

    2013 & \textbf{Hospitality Coordinator, Antaragni ‘13}
           \footnotesize{
               \begin{itemize}[leftmargin=*]
                   \item Oversaw a team of \textbf{150} students for a span of 8 months.
                   \item Chalked out the logistics for the hospitality and accommodation of \textbf{1500 students} who
                       participated in festival.
                   \item Made sure everything went smoothly during the four days of the Festival from 24$^{th}$
                       to 27$^{th}$ October.
               \end{itemize}
           }\\

    2013 & \textbf{Editor-in-chief, Vox Populi}, the campus newsletter. \\

    2012 & \textbf{Member, Gymkhana Review Committee}
           \footnotesize{
               \begin{itemize}[leftmargin=*]
                   \item The GRC was established to revamp the Students' Gymkhana of IIT Kanpur,
                       which is responsible for all the student activities of the institute.
                   \item I chaired the sessions on \textbf{Extended Orientation of Freshmen}.
                       And gave inputs to other academic, senate and activities sub-committees formed under the GRC.
               \end{itemize}
           }\\

    2012 & \textbf{Senator, Students' Senate}, IITK Y11 batch. \\

    2012 & \textbf{Secretary, English Literary Society}. \\

    2010 & Basic training in \textbf{Indian Classical Music}. (Prayag Sangeet Samiti Allahabad). \\

    2009 & Qualified for the semi-finals of \textbf{Outlook SpeakOut Debate}. \\

\end{tabular}

\section{CourseWork}

\begin{tabular}{>{\raggedleft}p{8cm}|p{8cm}}

    Theory of Computation (CS340) & (CS201) Discreet Mathematics \\
    Abstract Algebra (CS203A) & (CS202A) Logic for Computer Science \\
    Data Structure and Algorithms (CS210) & (CS345) Design and Analysis of Algorithms \\
    Operating Systems (CS330) & (CS335) Compilers \\
    Computer Networks (CS425) &  (CS628) Computer and Internet Security \\
    Computer Organization (CS220) & (CS350) Principles of Programming Languages \\
    Artificial Intelligence (CS365) & (CS251 \& CS252) Computing Laboratory \\
    Computational Methods in Engineering (ESO208A) & (CS300) Technical Communication \\
    Linear Algebra (MTH102) &  (MTH101) Multivariate Calculus \\
                     \\
\end{tabular}
\end{document}
