%----------------------------------------------------------------------------------------
%	PACKAGES AND OTHER DOCUMENT CONFIGURATIONS
%----------------------------------------------------------------------------------------

\documentclass[a4paper,10pt]{article} % Default font size and paper size

\usepackage{fontspec} % For loading fonts
\defaultfontfeatures{Mapping=tex-text}
\setmainfont[SmallCapsFont = Fontin SmallCaps]{Fontin} % Main document font

\usepackage{xunicode,xltxtra,url,parskip} % Formatting packages

\usepackage{marvosym} % Allows the use of symbols
\usepackage{scrpage2} % Provides headers and footers configuration

\usepackage[usenames,dvipsnames]{xcolor} % Required for specifying custom colors

% \usepackage[big]{layaureo} % Margin formatting of the A4 page, an alternative to layaureo can be \usepackage{fullpage}
% To reduce the height of the top margin uncomment: \addtolength{\voffset}{-1.3cm}

\usepackage{hyperref} % Required for adding links	and customizing them
\definecolor{linkcolour}{rgb}{0,0.2,0.6} % Link color
\hypersetup{colorlinks,breaklinks,urlcolor=linkcolour,linkcolor=linkcolour} % Set link colors throughout the document

\usepackage{tabularx,colortbl} % Advanced table configurations

\usepackage{titlesec} % Used to customize the \section command
\titleformat{\section}{\Large\scshape\raggedright}{}{0em}{}[\titlerule] % Text formatting of sections
\titlespacing{\section}{0pt}{3pt}{3pt} % Spacing around sections

% Custom arrangement for the columns
\usepackage[margin=2cm]{geometry}
\usepackage{array}

\begin{document}

\font\fb=''[cmr10]'' % Change the font of the \LaTeX command under the skills section

\titleformat{\section}{\large\scshape\raggedright}{}{0em}{}[\titlerule] % Section formatting

\pagestyle{scrheadings} % Print the headers and footers on all pages

% \addtolength{\voffset}{-0.5in} % Adjust the vertical offset - less whitespace at the top of the page
% \addtolength{\textheight}{3cm} % Adjust the text height - less whitespace at the bottom of the page
% \addtolength{\textwidth}{-2cm}


%----------------------------------------------------------------------------------------
% FOOTER SECTION
%----------------------------------------------------------------------------------------

\renewcommand{\headfont}{\normalfont\rmfamily\scshape} % Font settings for footer

\cofoot{
\addfontfeature{LetterSpace=20.0}\fontsize{10.5}{17}\selectfont % Letter spacing and font size

I-308, IIT Kanpur {\large\textperiodcentered} Kanpur {\large\textperiodcentered} UP 208016\\ % Your mailing address
{\Large\Letter} srijan.shetty@gmail.com \ {\Large\Telefon} +(91) 9005900383 % Your email address and phone number
}

%----------------------------------------------------------------------------------------
% HEADER SECTION
%----------------------------------------------------------------------------------------

{\centering
    {\addfontfeature{LetterSpace=20.0}\fontsize{36}{36}\selectfont\scshape
    Srijan R. Shetty}\\
}

%----------------------------------------------------------------------------------------
%   INTERESTS
%----------------------------------------------------------------------------------------
\section{Interests}

Programming Languages, Distributed Systems, Systems and Security, Web-Dev, NLP

%----------------------------------------------------------------------------------------
%	EDUCATION
%----------------------------------------------------------------------------------------
\section{Education}

\begin{tabular}{>{\raggedleft}p{2.2cm}p{12cm}r}

    \textsc{Current} & B. Tech in \textsc{Computer Science and Engineering} &   9.6/10.0\\
                     & \textbf{Indian Institute of Technology}, Kanpur\\

    \textsc{July 2011} & 12th Board, \textsc{CBSE} Board                    &   91.4\% \\
                       & \normalsize\textbf{The Emerald Heights International School}, Indore\\

    \textsc{July 2011} & 10th Board, \textsc{ICSE} Board                    &   95.4\% \\
                       & \normalsize\textbf{The Laurels School International}, Indore \\

\end{tabular}

%----------------------------------------------------------------------------------------
%	SCHOLARSHIPS AND CONFERENCES
%----------------------------------------------------------------------------------------
\section{Scholarships and Conferences}

\begin{tabular}{>{\raggedleft}p{2.2cm}p{14cm}}

    \textsc{Jan 2011} & Kishore Vaigyanik Protsahan Yojana (KVPY) Fellowship \\
    \textsc{Dec 2008} & FIITJEE Talent Reward Examination Scholarship \\
    \textsc{Dec 2009} & CSIR Programme on Youth for Leadership in Science (CPYLS camp),
                        held at Advanced Materials Research Institute, Bhopal\\

\end{tabular}

%----------------------------------------------------------------------------------------
%   SCHOLASTIC ACHIEVEMENTS
%----------------------------------------------------------------------------------------
\section{Scholastic Achievements}

\begin{tabular}{>{\raggedleft}p{2.2cm}p{14cm}}

    \textsc{2012-2013}   & \textbf{Academic Excellence Award}, IIT Kanpur. \\
    \textsc{2011-2012}   & \textbf{Academic Excellence Award}, IIT Kanpur. \\
    \textsc{Fall 2012}   & \textbf{Academic Mentor}, \textbf{Fundamentals of Computer Science (ESC101)}. \\
                         & \footnotesize{As an academic Mentor, I was entrusted with the responsibility of making sure
                            that the students under me were able to keep pace with instructor and aid them in understanding
                            the difficult concepts of the course. }\\
    \textsc{Jan 2010}    & Qualified for \textbf{Indian National Mathematics Olympaid} (INMO) \\
    \textsc{May 2011}    & Secured All India rank \textbf{1937} in Joint Entrance Exam (JEE) \\
    \textsc{Sep 2010}    & \textbf{AIR 17} in \textbf{Technothlon} (Techniche) conducted by Indian Institute of Technology Guwahati. \\
    \textsc{June 2010}   & Awarded \textbf{Top Scorer in English} by \textbf{Laurels School International} in ICSE 2009. \\
    \textsc{Dec 2009}    & Awarded \textbf{Overall Best Student} for displaying all round excellence. \\
    \textsc{Dec 2005}    & Awarded \textbf{Overall Best Student} for displaying all round excellence. \\
    \textsc{2000 - 2009} & Recipient of \textbf{Scholastic Excellence} awards for exemplary academic
                           performance throughout 2nd to 12th grade. \\

\end{tabular}

%----------------------------------------------------------------------------------------
%   SKILL SET
%----------------------------------------------------------------------------------------
\section{Skill Set}

\begin{tabular}{>{\raggedleft}p{2.2cm}p{14cm}}

    \textsc{Languages} %
            & \textbf{C}, \textbf{C++},\textbf{Python}, \textbf{JavaScript} \\
    \textsc{Web Dev} %
            & \textbf{HTML5}, \textbf{CSS}, \textbf{JavaScript}, \textbf{Flash}, \textbf{SQL},
              \textbf{AngularJS}, \textbf{Nodejs}, \textbf{NoSQL}\\
    \textsc{Tools} %
            & \textbf{Git}, \textbf{Shell Scripting}, \textbf{Prezi}, \textbf{Vim}, \textbf{LaTeX}\\

\end{tabular}

%----------------------------------------------------------------------------------------
%   WORK EXPERIENCE
%----------------------------------------------------------------------------------------

\section{Work Experience}

\begin{tabular}{p{2.2cm}|p{14cm}}
    \textsc{May-Jul 2014} & Research Intern at \textsc{Microsoft Research India}, Bangalore\emph{}\\
                          & \footnotesize{The major objective of this project was to optimize
                            \textbf{CScale for concurrency}. CScale provides a declarative programming model
                            using LINQ, it does all the heavy lifting of scaling distributed systems and
                            failure. The project also involved \textbf{failure testing and performance testing}
                            of the concurrency optimizations in real world scenarios.}\\
\end{tabular}

\begin{tabular}{p{2.2cm}|p{14cm}}
    \textsc{May-Jul 2013} & Software Development at \textsc{Aurus Networks}, Bangalore\emph{}\\
                          & \footnotesize{Using \textbf{HTML5, javascript, CSS3}, I ported the legacy
                            \textbf{Flash} based online media player of \textbf{CourseHub} - Aurus'
                            propriety MOOC offering. A new \textbf{XMPP} chat service for group chat
                            was also created and integrated with CourseHub.}\\
\end{tabular}

%----------------------------------------------------------------------------------------
%  COURSE PROJECTS
%----------------------------------------------------------------------------------------
\section{Course Projects}

\begin{tabular}{p{2.2cm}|p{14cm}}
    \textsc{Spring 2014} & JavaScript to MIPS Compiler, \textsc{\raggedright Compilers} \\
                         & \footnotesize{An end-to-end compiler for a subset of ECMAScript5.1
                            which compiled to MIPS was implemented in Python. No runtime support
                            was provided ergo, static types and type annotations were used instead.
                            The project was undertaken in the aegis of \textbf{Professor Subhajit Roy}.
                            \href{https://github.com/srijanshetty/javascript-compiler} {Git Repo}} \\
\end{tabular}

\begin{tabular}{p{2.2cm}|p{14cm}}
    \textsc{Spring 2014} & Hindi author attribution, \textsc{\raggedright Artificial Intelligence} \\
                         & \footnotesize{The first half of this project was a multi-class clustering problem,
                            wherein authors were to be clustered according to their writing style.
                            The features used for the same were, word frequency counts and collocation
                            frequencies.  The second half of the project was to classify the text snippets
                            into different authors using supervised learning techniques.
                            The project was supervised by \textbf{Professor Amitabha Mukherjee} and was
                            inherently difficult due to the unavailability of stemmers and POS taggers for Hindi.
                            \href{https://github.com/srijanshetty/author-attribution} {Git Repo}}\\
\end{tabular}

\begin{tabular}{p{2.2cm}|p{14cm}}
    \textsc{Fall 2013} & NachOS, \textsc{Operating Systems} \\
                       & \footnotesize{The coursework included implementation of system calls; process
                          management and scheduling; memory management schemes and page replacement
                          algorithms in \textbf{NachOS}. \textbf{Professor Mainak Chaudhuri}, was
                          the professor-in-charge}\\
\end{tabular}

\begin{tabular}{p{2.2cm}|p{14cm}}
    \textsc{Fall 2013} & P2P File Sharing and Streaming, \textsc{Computer Networks} \\
                       & \footnotesize{A Node.js based local file server and media playback server was implemented
                          to understand the underpinnings of Transport Layer Protocols, socket programming,
                          and P2P architecture.  The project was supervised by \textbf{Professor Dheeraj Sanghi}.
                          \href{https://github.com/srijanshetty/nodesock} {Git Repo}}\\
\end{tabular}

\begin{tabular}{p{2.2cm}|p{14cm}}
    \textsc{Fall 2013} & IP Spoofing, \textsc{Computer Networks} \\
                       & \footnotesize{Under the guidance of \textbf{Professor Dheeraj Sanghi}, different methodologies
                          of spoofing IP addresses and different uses of spoofing IPs were implemented and tested
                          in a secure environment.}\\
\end{tabular}

\begin{tabular}{p{2.2cm}|p{14cm}}
    \textsc{Fall 2013} & Metro Travel Plan(Prolog), \textsc{Principles of Programming Languages} \\
                       & \footnotesize{Implementation of shortest paths in Prolog on the Delhi Metro route.
                          The professor-in-charge was \textbf{Professor P. Kurur}}\\
\end{tabular}

\begin{tabular}{p{2.2cm}|p{14cm}}
    \textsc{Fall 2013} & Erlang Traffic Server, \textsc{Principles of Programming Languages} \\
                       & \footnotesize{A simple traffic server implemented in Erlang using message passing
                          concurrency and tail call recursion functional aspects of the language under the
                          guidance of \textbf{Professor P. Kurur}}\\
\end{tabular}

\begin{tabular}{p{2.2cm}|p{14cm}}
    \textsc{Spring 2013} & 8-bit General Purpose Computer on a FPGA, \textsc{Computer Organization} \\
                         & \footnotesize{The GPC had a load-store architecture and was written in System Verilog and
                            implemented on a \textbf{Xilinx Spartan 3 FPGA}.  The project was under the
                            supervision of \textbf{Professor Subhajit Roy}.  It had a limited, but powerful
                            Instruction Set Architecture which could implement recursive functions, jumps,
                            loops  other basic building blocks of a simple assembly language.
                            \href{https://github.com/srijanshetty/220_y11} {Git Repo}} \\
\end{tabular}

\begin{tabular}{p{2.2cm}|p{14cm}}
    \textsc{Spring 2013} & Conference Dates Web Crawler, \textsc{Computing Laboratory} \\
                         & \footnotesize{A python based web crawler which could obtain paper submission
                            deadlines of a provided conference name.  BeautifulSoup and urllib along
                            with simple pattern matching through regular expressions were used to implement crawling.
                            \href{https://github.com/srijanshetty/crawler} {Git Repo} } \\
\end{tabular}

\begin{tabular}{p{2.2cm}|p{14cm}}
    \textsc{Fall 2011} & "No situation is unique and certain moral principles can be applied across all situation",
                         \textsc{Introduction to Philosophy} \\
                       & \footnotesize{ An argumentative discourse proving that certain moral principles
                         can be applied throughout; by proving different situations have a certain
                         fundamental similarity.  Albeit this would require a gross simplification of
                         different situations.  The reviewing instructor was \textbf{Professor Vineet Sahu}} \\
\end{tabular}

%----------------------------------------------------------------------------------------
%  SIDE PROJECTS
%----------------------------------------------------------------------------------------
\section{Side Projects}

\begin{tabular}{p{2.2cm}|p{14cm}}
    \textsc{July 2014} & OARS\\
                       & \footnotesize{OARS is a \textbf{AngularJS} based front-end for the
                          courses offered in IIT Kanpur. It was created as a
                          hack to alleviate the pain of the students in using the
                          sluggish legacy course search. The data was obtained by
                          scraping the official data portal of IIT Kanpur, and the
                          project is now maintained by \textbf{Navya}
                          \href{https://github.com/navya/oars} {Git Repo} }\\

\end{tabular}

\begin{tabular}{p{2.2cm}|p{14cm}}
    \textsc{March 2014} & GYPH\\
                        & \footnotesize{GYPH or Get Your Personal Homepage
                           is a simple hassle-free homepage designer for students who are not
                           proficient in web technologies. It provides a WYSIWIG
                           interface for editing themes and downloading them for personal use.
                           \href{http://gyph2.herokuapp.com/} {Website} } \\
\end{tabular}

\begin{tabular}{p{2.2cm}|p{14cm}}
    \textsc{August 2013} & ShuffleRun\\
                         & \footnotesize{A hack for Yahoo HackU. ShuffleRun is a
                            web based music player which selects a track from the users library
                            according to her running speed.
                            \href{https://github.com/srijanshetty/ShuffleRun} {GitHub Repo} }\\
\end{tabular}

\begin{tabular}{p{2.2cm}|p{14cm}}
    \textsc{July 2013} & Website of Udghosh, annual sport-fest IIT Kanpur \footnotesize{\href{www.udghosh.org} {Webpage} } \\
\end{tabular}

\begin{tabular}{p{2.2cm}|p{14cm}}
    \textsc{Dec 2012} & Website of Hall 5, IIT Kanpur \footnotesize{\href{http://www.iitk.ac.in/hall5} {Webpage}} \\
\end{tabular}

\begin{tabular}{p{2.2cm}|p{14cm}}
    \textsc{Summer 2012} & Voicing, Archiving and Networking the Information \textsc{(VANI)}, IIT Kanpur\\
                           & \footnotesize{Vani is an \textbf{extensible} platform
                              made for the campus community consisting of Wiki, Forums,
                              Community search, Calendars etc.; built using \textbf{Drupal}
                              under the mentorship of \textbf{Professor Manindra Agarwal} and
                              \textbf{Professor T. V. Prabhakar} } \\
\end{tabular}

\begin{tabular}{p{2.2cm}|p{14cm}}
    \textsc{Summer 2012} & GNU/Linux Exploration, Programming Club, IIT Kanpur\\
                           & \footnotesize{ The project involved exploration of various facets of
                             \textbf{GNU/Linux} like the file system, process management,
                             memory management, shell interface etc. And was
                             undertaken under the aegis of \textbf{Programming Club, IIT Kanpur}
                             \href{https://docs.google.com/document/d/1ZHO9w36aoq3oaZBR4Um1AOmDfiTDAEgM6baQAu3icw4/edit?usp=sharing}
                             {Documentation} } \\
\end{tabular}

\begin{tabular}{p{2.2cm}|p{14cm}}
    \textsc{July 2013} & Projects on Github \\
                       & \footnotesize{\textbf{Dotfiles}: an opinionated work flow on Linux Systems.
                          \href{https://github.com/srijanshetty/dotfiles} {Repo} } \\
                       & \footnotesize{\textbf{Prezto}: a fork of the Prezto ZSH framework
                          \href{https://github.com/srijanshetty/prezto} {Repo}} \\
                       & \footnotesize {\textbf{oh-my-zsh}: a fork of the oh-my-zsh ZSH framework
                          \href{https://github.com/srijanshetty/oh-my-zsh} {Repo} } \\
                       & \footnotesize{\textbf{DS}: implementation of certain algorithms and data structures
                          \href{https://github.com/srijanshetty/DS} {Repo}} \\
\end{tabular}

%----------------------------------------------------------------------------------------
%   OTHER SKILLS
%----------------------------------------------------------------------------------------

\section {Social, Leadership and Artistic Skills}

\begin{tabular}{>{\raggedleft}p{2.2cm}p{14cm}}

    -   & Active blogger at \href{srijanshetty.quora.com} {srijanshetty.quora.com} \\

    -    & \textbf{Hacker, Navya, FOSS Group IIT Kanpur}\\
         & \footnotesize{Navya is the Free and Open Source Software Group at IIT
           Kanpur which educates the campus community about the FOSS mission and
           hacks on FOSS applications which would be of use to the students
           community \href{https://github.com/navya} {Github Repo}} \\

    2013-2014    & \textbf{Volunteer, English Proficiency Programme}\\
                 & \footnotesize{English Proficiency Programme is the brainchild
                    of \textbf{Professor Bhaskar Dasgupta}. The objective of the
                    programme is to impart a functional knowledge of English to
                    any interested student. I was a volunteer of the first leg
                    of the programme which helped under-privileged students.}\\

    2014 & \textbf{Head, Major Events and Competitions, Antaragni'14} , the annual cultural festival of
    IIT Kanpur. \\

    2013 & \textbf{Hospitality Coordinator, Antaragni ‘13}\\
         & \footnotesize{As a Hospitality Coordinator, I oversaw a team of 150 students responsible for
            the hospitality of the 1500 students who visit the campus for Antaragni, the annual cultural festival
            of IIT Kanpur which takes place in October every year.}\\

    2013 & \textbf{Editor-in-chief, Vox Populi}, the campus newsletter. \\
         & \footnotesize{Vox Populi is the campus newsletter of IIT Kanpur with a reach of more than 6000 students.
            As, the Editor-in-chief, I was responsible for ensuring impeccable quality of content in the newsletter and
            logistics of publishing the bi-monthly newsletter.}\\

    2012 & \textbf{Member Gymkhana Review Committee}\\
         & \footnotesize{The Gymkhana Review Committee was set up with the vision of revamping the Student's Gymkhana
            of IIT Kanpur, which is the body responsible for all extra curricular activities of the campus. As a
            member of the Gymkhana Review Committee, I chaired the sessions on \textbf{Extended Orientation of
            incoming freshmen students}, as a member of the committee, I gave valuable inputs to other academic,
            senate and activities reforms introduced by the committee.}\\

    2012 & \textbf{Senator, Students' Senate}, IITK Y11 batch. \\

    2012 & \textbf{Secretary, English Literary Society}. \\

    2010 & A year of schooling in \textbf{Indian Classical Music}. (Prayag Sangeet Samiti Allahabad). \\

\end{tabular}

\section{CourseWork}

\begin{tabular}{>{\raggedleft}p{8cm}|p{8cm}}

    Computer Organization (CS220) & Operating Systems (CS330) \\
    Compilers (CS335) & Computer Networks (CS425) \\
    Theory of Computation (CS3) & Computing Laboratory (CS251 \& CS252)\\
    Discreet Mathematics (CS201) & Logic for Computer Science (CS202A)\\
    Data Structure and Algorithms (CS210) & Design and Analysis of Algorithms (CS345)\\
    Principles of Programming Languages (CS350) & Computer and Internet Security (CS628)\\
    Artificial Intelligence (CS365) & Abstract Algebra (CS202B)\\
    Linear Algebra (MTH102) & Computational Methods in Engineering (ESO208A)\\
    Multivariate Calculus (MTH101) & Technical Communication(CS300)\\
                     \\
\end{tabular}
%----------------------------------------------------------------------------------------
%  SIDE PROJECTS
%----------------------------------------------------------------------------------------
\end{document}
